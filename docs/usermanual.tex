\documentstyle[12pt,supertabular, epsf]{article}
\title{XRFCOMBI\\
User Manual}
\author{M.Bos\\
University of Twente\\
POBox 217\\
7500 AE ENSCHEDE
NETHERLANDS}
\date{November 1999}
\begin{document}
\maketitle
\section{INTRODUCTION}
This manual describes the installation and use of the set of
programs described in the documents {\em aca1998.ps} and 
{\em xrfirreg.ps}. The user interface
is written in Tk/Tcl. Menu buttons give access to the various programs
and some utilities to simplify data entry. Data transfer between
the various programs is performed by temporary ascii-files that can be inspected
(and edited) by any plain ascii text editor.

The calculation methods {\bf XRFSMPLX}, {\bf XRFROUS} and\\
{\bf SMPLX/GICAL}  handle concentrations of
sample components in terms of chemical compounds directly and produces
results in terms of the concentrations of chemical compounds. All other
programs start with a conversion of the initial  estimate of the
sample composition to an estimated elemental composition, refine this
elemental composition and produce the result also as an elemental composition.

The choice of the program to use for calculating analytical results
should be governed by the guidelines given in the abovementioned papers.

Bug reports and suggestions for improvement of the programs can be sent
by e-mail to m.bos@ct.utwente.nl.

\section{INSTALLATION}
\subsection{Linux}
The software is available as a gzipped tar archive {\bf xrf011.tgz}
and can be unpacked in  a users directory by the command:
\begin{verbatim}
tar -xvzf xrf011.tgz
\end{verbatim}

To function the software needs an installed version of Tk/Tcl. The program
{\bf wish} of this package should be reachable via {\bf /usr/bin/wish}.
If this is not the place of
{\bf wish}, the first line of the script {\bf \~/combi/bin/xrfcombi}
should be adapted to point to the place of {\bf wish}.

The package {\em gnuplot} should be available if one wants to
use the plot option in the File menu of the calibration step
to display calibration lines.

The printing of results is routed via the standard unix printer of
the {\bf lpr} command. So attention has to be paid to the {\bf printcap} file.


\subsection{W95/W98}
The software is available as a zipped archive and can be unpacked
in a user directory. 
To function the software needs an installed version of Tk/Tcl for Windows.
The package {\em gnuplot} should be available if one wants to
use the plot option in the File menu of the calibration step
to display calibration lines.
The printing of results is routed via a utility program {\bf lprint.exe}
which copies the output to be printed in plain ascii to the port {\bf LPT1}.
If no printer is connected to this port the program will hang.


\section{USING THE PROGRAM}
Once installed the program can be started in its working directory
{\bf combi/bin} 
by typing {\bf xrfcombi} (Linux) or {\bf xrfgo} (Windows)
 after which the main menu bar  (see figure
\ref{mainmenu}) shows up.
\setlength{\unitlength}{1.0cm}
\begin{figure}[ht]
\begin{picture}(10,1)
\put(2,0)
{\setlength{\epsfxsize}{10.0cm}\leftline{\epsffile{mainmenu.eps}}}
\end{picture}
\caption{Main menu of XRFCOMBI}
\label{mainmenu}
\end{figure}

 The first thing to do now is to set
the spectrometer characteristics by clicking on the {\bf Spectrometer}
button with the left mouse button. Now a data-entry form 
{\bf SPECTROM.PARAM} appears (figure \ref{spectrom})
\setlength{\unitlength}{1.0cm}
\begin{figure}[ht]
\begin{picture}(7,4)
\put(3,0)
{\setlength{\epsfxsize}{7.0cm}\leftline{\epsffile{spectrom.eps}}}
\end{picture}
\caption{Entry form for spectrometer data}
\label{spectrom}
\end{figure}
 and all items of it should be filled
out by mouse clicking on the entry and typing the pertinent data
in the entries  box. When done the {\bf To File} button should
be clicked and the window can be dismissed.

Depending on the task that has to be performed one can now make a choice
with the buttons {\bf Simulate}, {\bf Calibrate} and {\bf Analyze}.
{\bf Help} is not implemented yet.

In the {\bf Simulate}, {\bf Calibrate} and {\bf Analyze} tasks
one can define samples or standard samples in terms of already defined
compounds. The definitions of these compounds are kept by the system
in the file {\bf complist.dat}. New entries in this list can only
be made via the {\bf Simulate} menu.

\subsection{Simulate}
The simulation menu is meant to calculate relative intensities for given
lines belonging to elements of a sample with known composition.
Clicking on the {\bf Simulate} button of the main menu bar produces
the simulation menu (figure \ref{simmenu}) via which composition data 
and the wanted lines can be given.
\setlength{\unitlength}{1.0cm}
\begin{figure}[ht]
\begin{picture}(14,1)
\put(0,0)
{\setlength{\epsfxsize}{14.0cm}\leftline{\epsffile{simmenu.eps}}}
\end{picture}
\caption{Bar for Simulation Menu}
\label{simmenu}
\end{figure}

\subsubsection{Compounds}
Now the first thing to do is to check whether the systems knows all
the compounds of which the sample has to be composed. This can be performed
by clicking the {\bf Compounds} button that produces two windows,
one named {\bf COMPOUNDS} that shows the list of all known compounds
and an entry form to define new compounds (see figure \ref{defcomps}).
\setlength{\unitlength}{1.0cm}
\begin{figure}[ht]
\begin{picture}(7,4)
\put(3,0)
{\setlength{\epsfxsize}{7.0cm}\leftline{\epsffile{defcomps.eps}}}
\end{picture}
\caption{Entry form for definition of Compounds}
\label{defcomps}
\end{figure}
 If the list contains all
the wanted compounds, the entry form can be dismissed. If not, this entry
form should be filled out starting with the name of the compound to be
defined in the
{\bf FORMULA/NAME}  box.  Next the density of the compound should be
entered in the density box. This quantity is not yet used in the 
calculations, so at the moment this value is not important,
but in the future it will be used to convert mass thickness to
normal thickness, so it is better to enter the correct value here.
Anyhow a value should be entered, otherwise the format of the 
complist.dat file will be disturbed.

In this release all compounds should be defined in terms of number
of atoms of the elements present in the compound, so the radiobutton
{\bf AT} should be on(red). If {\bf WT} is chosen the program refuses
to accept the entry for now.

Entering the number of different elements present in the compound
in the box {\bf NR\_ELEMENTS} opens a number of entries
in which the symbols of these elements should be entered, followed
by the number of atoms of the element in the compound. When the
information is entered correctly, then click the {\bf OK-ADD to FILE} button
and proceed to the next compound to be defined, overwriting the
info in the {\bf FORMULA/NAME} and density box. Clicking in the
box {\bf NR\_ELEMENTS} removes the old element/number of atoms information
and entering a new number of elements for the new compound creates
fresh entries for the symbol/ number of atoms items.

When the last new compound has thus been defined and
been entered in the list 
 with {\bf OK-ADD TO FILE},
the entry form can be dismissed.

\subsubsection{Sample}
Clicking the {\bf Sample} button in the {\bf SIMULATION MENU} starts
the definition of the composition of the sample for which
the intensities have to be calculated. This action opens two windows,
the {\bf COMPOUND SELECTION} window (figure \ref{compsel}) and
the {\bf DEFINE SAMPLE} window (figure \ref{defsmpl}).
\setlength{\unitlength}{1.0cm}
\begin{figure}[ht]
\begin{picture}(4,12)
\put(5,0)
{\setlength{\epsfxsize}{4.0cm}\leftline{\epsffile{compsel.eps}}}
\end{picture}
\caption{List box with compounds to select}
\label{compsel}
\end{figure}
Continue by entering the requested data for {\bf NAME} and {\bf THICKNESS}
\setlength{\unitlength}{1.0cm}
\begin{figure}[ht]
\begin{picture}(7,5)
\put(3,0)
{\setlength{\epsfxsize}{7.0cm}\leftline{\epsffile{defsmpl.eps}}}
\end{picture}
\caption{Entry form for definition of Sample}
\label{defsmpl}
\end{figure}
in their boxes. Next click on the names of those compounds listed
 in the {\bf COMPOUND
SELECTION} listbox that should be included in the sample. These compounds 
will appear automatically as entries in the {\bf DEFINE SAMPLE}
entry form. When all necessary compounds have been selected,
dismiss the {\bf COMPOUND SELECTION} box. Then adjust the
fractions of the compounds in the {\bf DEFINE SAMPLE} form to the right
values either by overwriting the fractions shown or just entering them
in their boxes. Now click on the {\bf OK ADD TO FILE} button. If you plan
to do calculations on a sample with the same compounds but with  different
contents or with a different thickness, do not dismiss this entry form to be able to use
it later without having to select all compounds again.

The boxes {\bf HOMOGENEOUS}, {\bf PARTICULATE}, {\bf VOID FRACTION}
and {\bf VOID SIZE}  are not yet used in the program, so don't bother
with them.

Now the lines for which the intensities haven to be calculated
should be entered. Start this process by clicking
on the {\bf Lines} button of the {\bf SIMULATION MENU}. This opens
the {\bf DEFINE LINES} entry form (figure \ref{deflines}).
\setlength{\unitlength}{1.0cm}
\begin{figure}[ht]
\begin{picture}(7,11)
\put(4,0)
{\setlength{\epsfxsize}{7.0cm}\leftline{\epsffile{deflines.eps}}}
\end{picture}
\caption{Entry form for definition of lines}
\label{deflines}
\end{figure}
This entry form is also used in other places in the program where more data
is needed than for the simulations that we are dealing with now,
so not all entries need to be filled. It suffices
to enter the data on {\bf Element} (symbol only),
{\bf Line} (line type ka, kb1, kb2, la1, la2, lb1 etc.),
{\bf kV} (the voltage the x-ray tube is operated on in kV when
the line is measured) and to set the radiobutton {\bf FILTER ON}. When
the latter button is red the calculations are performed for the
presence of an aluminium filter of 300 $\mu$m in the primary beam (we have
only this one filter).

Pressing {\bf OK ADD TO LIST} enters this line into the list
and when all lines have thus been entered, pressing {\bf CLOSE FILE}
writes this list to a diskfile that is used in the calculations. 
It is possible that on clicking {\bf OK ADD TO LIST} that an alert box
{\bf WRONG INPUT?} appears. In that case just click on {\bf Do it anyway}.
After all lines have been entered the form can be dismissed.

At this stage sufficient data has been entered to start the
simulation calculations by clicking on the {\bf Simulate} button
of the {\bf SIMULATION MENU}. As long as this button remains sunken
the calculations are in progress. When it returns to its raised
state the calculations are ready and the results can be inspected
by openening the {\bf Viewer} dropdown menu on the {\bf SIMULATION
MENU} bar and clicking on the {\bf View Results Simulation} item.
This dropdown menu also gives the opportunity to inspect
all data files used for the calculation to check for entry errors.
The results can also be printed by opening the {\bf Print}
dropdown menu and choosing {\bf Print results}.

If the {\bf DEFINE SAMPLE} entry form was not dismissed earlier
it is now possible to repeat the calculations for a different
composition or thickness just by changing the pertinent data
in this entry form, writing it to file by clicking on {\bf OK ADD TO FILE}
button of this {\bf DEFINE SAMPLE MENU} form and clicking the {\bf Simulate}
button again, followed by Viewing or Printing.

When done with all the needed simulations, dismiss the {\bf DEFINE
SAMPLE} entry form and dismiss the {\bf SIMULATION MENU} by clicking
on the {\bf Dismiss Sim. Menu} button on its bar.

\subsection{Calibrate}
The button {\bf Calibrate} opens the menu bar  for the calibration
process (see figure \ref{calmenu}).
\setlength{\unitlength}{1.0cm}
\begin{figure}[ht]
\begin{picture}(14,1)
\put(0,0)
{\setlength{\epsfxsize}{14.0cm}\leftline{\epsffile{calmenu.eps}}}
\end{picture}
\caption{Menu bar for calibration}
\label{calmenu}
\end{figure}

Data is needed on the composition of the standard samples
used for the calibration and on the  measured fluorescence lines and
their intensities. Entry forms for this data can be opened
by clicking on the buttons {\bf Define Sample} (the compounds 
composing the sample), {\bf Enter Concs} (the fractions  of these
compounds in the standard), {\bf Define Lines} (the nature of
the lines used in the calibration) and {\bf Enter intensities}
(the measured intensities for all measured lines for all samples).

The data should be entered in a specific order: {\bf Define Sample} -$>$
{\bf Enter Concs} -$>$ {\bf Define Lines} -$>$ {\bf Enter intensities}.

\subsubsection{Define Sample}
Clicking on the {\bf Define Sample} button of the {\bf CALIBRATION MENU} bar
opens the listbox {\bf COMPOUND SELECTION} (see figure \ref{compsel} containing all the compounds that
 the system knows of and a box {\bf SAMPLE COMPOUNDS} that shows the
compounds entered thus far (see figure \ref{sampcomp}). 
\setlength{\unitlength}{1.0cm}
\begin{figure}[ht]
\begin{picture}(7,4)
\put(3,0)
{\setlength{\epsfxsize}{7.0cm}\leftline{\epsffile{sampcomp.eps}}}
\end{picture}
\caption{Window showing compounds in sample}
\label{sampcomp}
\end{figure}
Compounds are entered into the sample by clicking on their name 
in the listbox {\bf COMPOUND SELECTION}. There is no error correction
possibility. On an error just dismiss the {\bf SAMPLE COMPOUNDS} box
and start again by clicking on the {\bf Define Sample} button of the
{\bf CALIBRATION MENU}. When all compounds that are needed have been entered,
dismiss both boxes.

\subsubsection{Enter Concs}
The entry of the composition of the various calibration samples is started
by clicking on the {\bf Enter Concs} button of the {\bf CALIBRATION MENU} bar.
This action opens an entry form {\bf ENTER CONCS} (see figure \ref{entcon})
\setlength{\unitlength}{1.0cm}
\begin{figure}[ht]
\begin{picture}(7,4)
\put(3,0)
{\setlength{\epsfxsize}{7.0cm}\leftline{\epsffile{entcon.eps}}}
\end{picture}
\caption{Entry form for concentrations of compounds in sample}
\label{entcon}
\end{figure}
and a table {\bf TABLE OF CONCS} (see figure \ref{tblconcs}) that gives
an overview of the composition of the standard samples entered thus far.
\setlength{\unitlength}{1.0cm}
\begin{figure}[ht]
\begin{picture}(10,5)
\put(2,0)
{\setlength{\epsfxsize}{10.0cm}\leftline{\epsffile{tblconcs.eps}}}
\end{picture}
\caption{table of entered sample compositions}
\label{tblconcs}
\end{figure}
The entry form should be filled with a sample identification 
({\bf Sample ID}), the massthickness in g/cm2 of the sample and
the fractions of its constituents. If all entries have been entered
correctly the {\bf Enter} button should be pressed to copy
the data for this particular sample to the table where it appears
automatically. When all standard samples have been treated, the
{\bf Done} button of the entry form can be clicked.

The next step is to check the contents of the table {\bf TABLE OF CONCS}.
If there are mistakes, then correct them by clicking on the
pertinent box and type the correct value. When all is ok, then
click on the {\bf OK} button of the table. 

\subsubsection{Define Lines}
With the {\bf Define Lines} button the same entry form as used
in the Simulate menu is opened to indicate which X-Ray Fluorescence Lines
haven been measured (see figure \ref{deflines}). Now, however, we are
dealing with specific measurements, so all information should be
entered correctly. In our laboratory we we either measure with a large 
mask or a small one, or we measure in cups closed with a mylar film.
The choice can be indicated by clicking on the correct radiobutton.

Furthermore we have the choice between four analyzing crystals. The 
choice is indicated to the system by clicking on one of the
radiobuttons {\bf LIF-200}, {\bf GE}, {\bf PE} or {\bf PX1}.

The only thing the system does with this data is to mark calculated
calibration constants as belonging to these specific conditions 
(in the file {\bf listio.dat}, so that
in the analyzing phase the correct calibration constants can be found
for a given set of measurement conditions. So if you don't find
your wanted conditions concerning masks and analyzing crystals
here, just pick one and be consistent in using it.

\subsubsection{Enter intensities}
The final thing to do before the calculations can be
started is to enter the measured intensities for all standard samples.
The process is started by clicking on the {\bf Enter intensities} button
of the {\bf CALIBRATION MENU} bar. This opens and entry form
{\bf ENTER INTENSITIES} in which the intensities measured for the
indicated sample should be given (see figure \ref{entint}).
\setlength{\unitlength}{1.0cm}
\begin{figure}[ht]
\begin{picture}(7,5)
\put(2,0)
{\setlength{\epsfxsize}{7.0cm}\leftline{\epsffile{entint.eps}}}
\end{picture}
\caption{Entry form for measured intensities}
\label{entint}
\end{figure}
Clicking
on the {\bf Enter} button of this form copies the data to the
{\bf TABLE OF INTENSITIES} (figure \ref{tblints}). 
\setlength{\unitlength}{1.0cm}
\begin{figure}[ht]
\begin{picture}(10,4)
\put(1,0)
{\setlength{\epsfxsize}{10.0cm}\leftline{\epsffile{tblints.eps}}}
\end{picture}
\caption{Table of measured intensities}
\label{tblints}
\end{figure}
If all  standard
samples have been treated the entry box shrinks and can be dismissed.

After checking the contents of the {\bf TABLE OF INTENSITIES} and
correcting the mistakes, this table can be dismissed by clicking 
its {\bf OK} button. The data is now written to a file which is used
in the calibration calculations.

\subsubsection{Results}
The calibration results can now be obtained by clicking on 
the {\bf Results} menubutton of the {\bf CALIBRATION MENU} bar.
This opens a pull down menu with which the right type of calculations
 can be started or  the
results can be inspected and printed. The button {\bf Normal calibration}
starts a calibration procedure that uses the normal relative intensities
based on the intensities of the pure elements. With the button
{\bf Gi-factor calibration} the calibration method from the paper
"Non-destructive analysis of small irregularly shaped homogeneous
samples by X-ray fluorescence spectrometry", M.Bos, et.al. in
Anal.Chim.Acta, is carried out. It is very well possible
to perform the calibration calculation in both ways. The results
are kept in different files, i.e. {\bf listio.dat} resp. {\bf listgi.dat}
the list of calibration constants ({\bf listio.dat}) that the
system uses in the analyzing phase.

\subsubsection{File}
The {\bf CALIBRATION MENU} bar has a {\bf File} menubutton that gives
access to some maintenance functions via a dropdown menu.
The last item on this menu {\bf Quit calibration} closes the calibration
menu bar. All files used in the calibration calculations can be archived
by clicking on the {\bf Save Calib Data} entry of this menu.
The entry {\bf Load Calib Data} restores archived calibration
to the standard files, which then can be edited by hand if necessary
so that a recalculation of calibration constants can be performed.

Finally the calibration constants can be inspected graphically
via the gnuplot program by clicking on the {\bf Plot Calib Data} entry
of the {\bf File} menubutton.

\subsection{Analyze}
The button {\bf Analyze} on the main menu bar (figure \ref{mainmenu})
opens the {\bf ANALYZE SAMPLES} menu bar (see figure \ref{analmenu}).
\setlength{\unitlength}{1.0cm}
\begin{figure}[ht]
\begin{picture}(10,1)
\put(1,0)
{\setlength{\epsfxsize}{10.0cm}\leftline{\epsffile{analmenu.eps}}}
\end{picture}
\caption{Analyze menu bar}
\label{analmenu}
\end{figure}
This bar shows a number of buttons that give access to the entry forms
that should be filled out to enter data on the qualitative composition
of the sample ({\bf Compounds in Sample}), the quantitative composition
of the sample for those components of which the concentration
is known together with the indication of the quantities
to be calculated from the measurements, i.e. unknown concentrations
and maybe the massthickness of the sample ({\bf Enter Concs}),
the definition of the lines that were used in the measurements 
({\bf Define Lines}) and the intensities of the measured lines ({\bf Enter
intensities}).
These entries should be dealt with in the order given.

The {\bf Results} button gives access  to the various calculation
methods and with the {\bf File} button data that has been entered
can be archived or saved data can be retrieved and/or edited.

Details for the various entry forms are given below.

\subsubsection{Compounds in sample}
The first thing to do in an analyzing session is clicking
 on the {\bf Compounds in sample} button of the {\bf ANALYZE SAMPLES}
menu bar and this  produces the same {\bf COMPOUNDS SELECTION}
 listbox with compounds from which the
samples can be composed as was used in the calibration menu 
(figure \ref{compsel}). Clicking on the names of the wanted compounds
enters them in the {\bf SAMPLE COMPOUNDS} table (see figure \ref{smplcomps}).
\setlength{\unitlength}{1.0cm}
\begin{figure}[ht]
\begin{picture}(7,4)
\put(3,0)
{\setlength{\epsfxsize}{7.0cm}\leftline{\epsffile{smplcomps.eps}}}
\end{picture}
\caption{Table of sample compounds}
\label{smplcomps}
\end{figure}
Both windows should be dismissed before continuing.

\subsubsection{Enter Concs}
The {\bf Enter Concs} button of the {\bf CALIBRATION MENU} bar
opens an entry form and a table. The entry form {\bf ENTER CONCS}
shows entries for the sample identification, the massthickness
and the concentration of the constituents of the sample
 (see figure \ref{concs}).
\setlength{\unitlength}{1.0cm}
\begin{figure}[ht]
\begin{picture}(10,6)
\put(2,0)
{\setlength{\epsfxsize}{10.0cm}\leftline{\epsffile{concs.eps}}}
\end{picture}
\caption{Entry form for sample composition/unknowns}
\label{concs}
\end{figure}
Unknown concentrations or an unknown
massthickness should be given as  question marks. Concentrations
that are known (i.e. the Li2B4O7 content) should be given as fractions
and are kept fixed during the calculations.

When this form has been completed for a sample to be analyzed the button
{\bf Enter} of this form should be pressed to copy the data to the
{\bf TABLE OF CONCS}. As soon as the data of the last sample has been entered
this form can be dismissed by clicking on its {\bf Done} button.

The next step is to inspect the {\bf TABLE OF CONCS} for errors, correcting
them if there are any and copying them to a system file used
in the calculations by clicking on its {\bf OK} button.

\subsubsection{Define Lines}
The definition  of the characteristics of the measured lines proceeds
exactly as in the calibration section. It is stressed here that
entering multiple lines for one element is only possible
for the calculation methods that use the simplex algorithm ({\bf XRFSMPX}
and {\bf SMPLX/GICAL}. The other calculation method don't converge
if multiple lines for one element are used.

\subsubsection{Enter intensities}
The procedure to enter the measured intensities is also the same
as in the calibration menu.

\subsubsection{Results}
The button {\bf Results} of the {\bf ANALYZE SAMPLE} menu bar gives access
to the various calculation methods as described in the Anal.Chim.Acta papers.
Choose the wanted calculation
method. The calculation results are available when the {\bf Results}
button returns from its raised state to the flat state and then can
be inspected with the {\bf View} item of the {\bf Results} menu
or printed with the {\bf Print} item.

\subsubsection{File}
The {\bf File} menu button of the {\bf ANALYZE SAMPLE} menu bar
gives access to  utilities for editing, storing and retrieving
the system files that are used in the calculation of the results.

The {\bf Quit} item dismisses the {\bf ANALYZE SAMPLE} menu bar.
 

 

\end{document}

